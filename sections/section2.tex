%! TEX root = ../main.tex

\section{Methodology}%
\label{sec:methodology}%

All figures, tables, and numbers in the manuscript which are subject
to change when the analysis data changes are computed using
\textsc{python} script and saved into auxilliary \LaTeX files. This
document illustrates the use of these variables (such as in the
sentence on the versions of \textsc{matplotlib} above) as well as the
figure layout options.
The necessary files are created with comments via the
\textsc{template.org} file in the GitHub repository. More information
on the usage can be found there.

\Cref{fig:hlines_demo} showcases the default figure
layout. This is a \textsc{pgf} figure, meaning that it is written in
\LaTeX commands and created at document compile time. Notice the
infinite resolution thanks to vector graphics and that the font in the
figure matches the one used in the rest of the document. The figure
spans exactly one column.

In \Cref{fig:matplotlib_release_dates}, we can see the use of a double-column \textsc{pgf} figure.
\Cref{fig:solution_space} shows a square figure. See the
documentation for implementation details.

The supplied \textsc{Makefile} contains some convenience functions for
compiling and sharing the document. 

The rest of this document is padded with Lorem Impsum.

\lipsum

\begin{table}[t]
\caption{An automatically generated table of \textsc{matplotlib} versions.}
\label{tab:matplotlib_versions}
\begin{tabular}{lr}
	\toprule
	Version & Date \\
	\midrule
v1.4.0 & 2014-08-26\\
v1.4.1 & 2014-10-18\\
v1.4.2 & 2014-10-26\\
v1.4.3 & 2015-02-16\\
v1.5.0 & 2015-10-29\\
v1.5.1 & 2016-01-10\\
v1.5.2 & 2016-07-03\\
v1.5.3 & 2016-09-09\\
v2.0.0 & 2017-01-17\\
v2.0.1 & 2017-05-02\\
v2.0.2 & 2017-05-10\\
v2.1.0 & 2017-10-07\\
v2.1.1 & 2017-12-10\\
v2.1.2 & 2018-01-18\\
v2.2.0 & 2018-03-06\\
v2.2.1 & 2018-03-16\\
v2.2.2 & 2018-03-17\\
v2.2.3 & 2018-08-10\\
v3.0.0 & 2018-09-18\\
v3.0.1 & 2018-11-10\\
v3.0.2 & 2018-11-10\\
v3.0.3 & 2019-02-26\\
v2.2.4 & 2019-02-26\\
	\bottomrule
\end{tabular}

\end{table}
\lipsum[3-5]

%% Local Variables:
%% tex-main-file: "../main.tex"
%% End:
