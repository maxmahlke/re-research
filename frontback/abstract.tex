The \textsc{Re: Research} publication template is meant to simplify
the writing of scientific manuscripts by creating a link between the
data analysis, done with the \textsc{python} programming language, and
the manuscript creation, built around the \LaTeX\xspace files. The premise is
that any figure, table, or number given in the manuscript which is
derived from your data should be placed in the manuscript by your
code. This removes the necessity of keeping the manuscript up-to-date
with changes in the input data. An additional bonus is an increase of
reproducible research: by publishing the scripts which compute the
manuscript content, anyone (including the original author) can verify
its derivation.
The template further provides a complete setup of \textsc{pgf}
integration for figure creation with the \textsc{matplotlib} package,
which allows to create vectorized figures utilizing the same font as
the rest of the document.
